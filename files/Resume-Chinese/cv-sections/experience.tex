%----------------------------------------------------------------------------------------
%	SECTION TITLE
%----------------------------------------------------------------------------------------

\cvsection{个人经历}

%----------------------------------------------------------------------------------------
%	SECTION CONTENT
%----------------------------------------------------------------------------------------

\begin{cventries}

%------------------------------------------------

\cventry
{计算机视觉和模式识别硕士} % Job title
{北邮模式识别和智能系统实验室} % Organization
{中国-北京} % Location
{2014年 9月 - 2017年 3月} % Date(s)
{ % Description(s) of tasks/responsibilities
\begin{cvitems}
\item {导师:\href{http://www.pris.net.cn/introduction/teacher/dengweihong}{邓伟洪} (http://www.pris.net.cn/introduction/teacher/dengweihong)。从事生物识别,物体检测,三维模型分析相关的机器学习工作,有深度学习经验。}
\end{cvitems}
}
%------------------------------------------------

\cventry
{机器学习研究员和软件开发者} % Job title
{OpenCV和Google开源社区} % Organization
{中国-北京} % Location
{2015年 4月 - 2016年 9月} % Date(s)
{ % Description(s) of tasks/responsibilities
\begin{cvitems}
\item {企业导师:\href{https://www.linkedin.com/in/stefano-fabri-16a73748}{Stefano Fabri}和\href{https://www.linkedin.com/in/manuele-tamburrano-b82384a5?authType=name&authToken=Di5p&trk=prof-sb-browse_map-name}{Manuele Tamburrano}。受Google Summer of Code 2015、2016项目资金支持。 维护tiny-dnn项目并开发3D物体检测和姿态估计API。\href{https://github.com/tiny-dnn/tiny-dnn}{tiny-dnn}的6名核心开发成员之一。\\
下面是一些项目演示的超链接:\href{https://www.youtube.com/watch?v=Mc20rTYdXTE}{3D Object Multi-task Learning}、\href{https://drive.google.com/open?id=0B-RYa1FDOrYXVUEzcG1mdnl5a3M}{tiny-dnn on iOS}
}
\end{cvitems}
}

%------------------------------------------------

\cventry
{软件开发者} % Job title
{阿里巴巴} % Organization
{中国-北京} % Location
{2015年 6月 - 2015年 7月} % Date(s)
{ % Description(s) of tasks/responsibilities
\begin{cvitems}
\item {2015年天池大数据竞赛用户购买倾向预测,全球排名第68(共1500个参赛队)。}
\end{cvitems}
}

%------------------------------------------------

\cventry
{软件开发者} % Job title
{WINE项目} % Organization
{中国-北京} % Location
{2015年 5月 - 2015年 6月} % Date(s)
{ % Description(s) of tasks/responsibilities
\begin{cvitems}
\item {从开源字符库到微软闭源字符库的匹配,使用我的2015奶奶CCBR会议论文local PCA filter作为基础算法。}
\end{cvitems}
}

%------------------------------------------------

\end{cventries}
